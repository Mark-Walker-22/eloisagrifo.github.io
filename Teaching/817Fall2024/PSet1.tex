\documentclass[11pt]{article}
\usepackage[margin=1in]{geometry}
\usepackage{amsmath,amsfonts,amssymb,amsthm,enumerate}
\usepackage[]{graphicx}
\usepackage{color,subfigure}
\usepackage{multicol}
\usepackage{float}
\usepackage[all]{xypic}
\usepackage[colorlinks=true,citecolor=cyan,linkcolor=magenta]{hyperref}
\usepackage{colonequals}

\usepackage{fancyhdr, lastpage}
\pagestyle{fancy}
\fancyfoot[C]{{\thepage} of \pageref{LastPage}}



\DeclareMathOperator{\mSpec}{mSpec}
\DeclareMathOperator{\Spec}{Spec}
\DeclareMathOperator{\Ass}{Ass}
\DeclareMathOperator{\Supp}{Supp}
\DeclareMathOperator{\height}{height}
\DeclareMathOperator{\Hom}{Hom}
\DeclareMathOperator{\ann}{ann}
\DeclareMathOperator{\End}{End}
\DeclareMathOperator{\coker}{coker}
%\DeclareMathOperator{\ker}{ker}
\DeclareMathOperator{\rank}{rank}
\DeclareMathOperator{\im}{im}
\DeclareMathOperator{\M}{M}
\DeclareMathOperator{\Tor}{Tor}
\DeclareMathOperator{\id}{id}
\DeclareMathOperator{\ch}{char}
\DeclareMathOperator{\Aut}{Aut}
%\DeclareMathOperator{\dim}{dim}

\DeclareMathOperator{\lcm}{lcm}

\def\ra{\rightarrow}
\newcommand{\m}{\mathfrak{m}}
\newcommand{\C}{\mathbb{C}}
\newcommand{\Q}{\mathbb{Q}}
\newcommand{\Z}{\mathbb{Z}}
\newcommand{\R}{\mathbb{R}}
\newcommand{\N}{\mathbb{N}}
\newcommand{\ov}[1]{\overline{#1}}

\def\ov#1{\overline{#1}}


\title{}
\date{\vspace{-0.5in}}

\makeatletter
\g@addto@macro\@floatboxreset\centering
\makeatother

\theoremstyle{definition}
\newtheorem{problem}{Problem}


\begin{document}

\thispagestyle{fancy}
\pagestyle{fancy}
\rhead{UNL $\mid$ Fall 2024}
\lhead{Introduction to Modern Algebra I}

\vspace{3em}

\begin{center}
	{\LARGE Problem Set 1 \\}
	Due Wednesday, September 4
\end{center}

\

\noindent
{\bf Instructions:}
You are encouraged to work together on these problems, but each student should hand in their own final draft, written in a way that indicates their individual understanding of the solutions. Never submit something for grading that you do not completely understand. You cannot use any resources besides me, your classmates, and our course notes.


I will post the .tex code for these problems for you to use if you wish to type your homework. If you prefer not to type, please  {\em write neatly}. As a matter of good proof writing style, please use complete sentences and correct grammar. You may use any result  stated or proven in class or in a homework problem, provided you reference it appropriately by either stating the result or stating its name (e.g. the definition of ring or Lagrange's Theorem). Please do not refer to theorems by their number in the course notes, as that can change.


\

\begin{problem}
Prove that every group of order $4$ is abelian. Your proof should only use the definition of a group. 
\end{problem}

\begin{problem}
Let $G$ be a group and $x \in G$ any element. 
Recall that $|x|$ denotes the {\em order} of $x$, defined to be the least integer $n \geqslant 1$ such that $x^n = e$; if no such integer exists, we say $|x| = \infty$.
Also, let $|G|$ denote the cardinality of $G$; note that $|G|$ is an element of $\{1, 2, 3, \cdots \} \cup \{\infty\}$.

\begin{enumerate}[(a)]

\item Prove that if $|x| = n$, then $e, x, \dots, x^{n-1}$ are all distinct elements of $G$.

\item Prove that if $|x| = \infty$, then $x^i \neq x^j$ for all integers $i \neq j$. 

\item Conclude that $|x| \leqslant |G|$ in all cases. 
\end{enumerate}
\end{problem} 



\begin{problem}
	A group $G$ is called {\it cyclic} if it is generated by a single element. 
	
	\begin{enumerate}[(a)]
		\item Prove that any cyclic group is abelian. 
		
		\noindent Note: your proof will be very short, as you can use the fact that $x^ix^j = x^{i+j}$ without proof.

		\item Prove that $(\mathbb Q, +)$ is not a cyclic group.
	\item Prove that $\operatorname{GL}_2(\mathbb Z_2)$ is not cyclic.
	\end{enumerate}
\end{problem}


\begin{problem}
For $n \geqslant 2$, prove that $S_n$ is generated by $(12)$ and the $n$-cycle $(12 \cdots n)$.

\noindent Note: If you are unsure which formulas about permutations require proof, please ask.
\end{problem}



\begin{problem}
Suppose the cycle type of $\sigma \in S_n$ is $m_1, m_2, \ldots, m_k$. Recall this means that $\sigma$ a product of disjoint cycles of lengths $m_1, m_2, \ldots, m_k$. Prove that $|\sigma| = \lcm (m_1, \ldots, m_k)$. 
\end{problem}


\end{document}