\documentclass[11pt]{article}
\usepackage[margin=1in]{geometry}
\usepackage{amsmath,amsfonts,amssymb,amsthm}
\usepackage{enumerate}
\usepackage[]{graphicx}
\usepackage{color,subfigure}
\usepackage{multicol}
\usepackage{float}
\usepackage[all]{xypic}
\usepackage{hyperref}
\usepackage{colonequals}
\usepackage{mathrsfs} 

\usepackage{fancyhdr, lastpage}
\pagestyle{fancy}
%\fancyfoot[C]{{\thepage} of \pageref{LastPage}}

\setlength{\itemsep}{10em}


\DeclareMathOperator{\mSpec}{mSpec}
\DeclareMathOperator{\Spec}{Spec}
\DeclareMathOperator{\Ass}{Ass}
\DeclareMathOperator{\Supp}{Supp}
\DeclareMathOperator{\height}{height}
\DeclareMathOperator{\Hom}{Hom}
\DeclareMathOperator{\ann}{ann}
\DeclareMathOperator{\End}{End}
\DeclareMathOperator{\coker}{coker}
%\DeclareMathOperator{\ker}{ker}
\DeclareMathOperator{\rank}{rank}
\DeclareMathOperator{\im}{im}
\DeclareMathOperator{\M}{M}
\DeclareMathOperator{\Tor}{Tor}
\DeclareMathOperator{\id}{id}
\DeclareMathOperator{\ch}{char}
\DeclareMathOperator{\Aut}{Aut}
\DeclareMathOperator{\PO}{\mathbf{PO}}
\DeclareMathOperator{\Ch}{Ch}
\newcommand{\Ob}{\mathrm{Ob}}
\newcommand{\Set}{\mathbf{Set}}
%\DeclareMathOperator{\dim}{dim}


\DeclareMathOperator{\Ext}{Ext}
\DeclareMathOperator{\injdim}{injdim}

\def\ra{\rightarrow}
\newcommand{\m}{\mathfrak{m}}
\newcommand{\C}{\mathbb{C}}
\newcommand{\Q}{\mathbb{Q}}
\newcommand{\Z}{\mathbb{Z}}
\newcommand{\R}{\mathbb{R}}
\newcommand{\N}{\mathbb{N}}
\newcommand{\ov}[1]{\overline{#1}}


\def\ov#1{\overline{#1}}


\title{}
\date{\vspace{-0.5in}}

\makeatletter
\g@addto@macro\@floatboxreset\centering
\makeatother

\theoremstyle{definition}
\newtheorem{problem}{Problem}


\begin{document}

\thispagestyle{fancy}
\pagestyle{fancy}
\rhead{UNL}
\lhead{Homological Algebra}

\vspace{3em}

\begin{center}
	{\LARGE Problem Set 4}\\
	Due Friday, December 1
\end{center}


\


\noindent
Turn in {\bf 5} of the following problems. Slightly more challenging problems are indicated by $(\star)$ .

%\
%
\noindent
%{\bf Instructions:}
%You are encouraged to work together on these problems, but each student should hand in their own final draft, written in a way that indicates their individual understanding of the solutions. Never submit something for grading that you do not completely understand. You cannot use any resources besides me, your classmates, and our course notes.

\vspace{2em}




\begin{problem}
	Let $R=k[x,y]$, where $k$ is a field, let $Q = \textrm{frac}(R)$ be the fraction field of $R$. We are going to show that the $R$-module $M = Q/R$ is divisible but not injective.
	\begin{enumerate}[a)]
		\item Show\footnote{If you know about regular sequences, this is easy to justify. But we aren't assuming anyone has seen regular sequences, so the challenge here is to give a clear, easy justification without invoking anything about regular sequences; though it's certainly ok to say the word regular.} that if $ax + by = 0$ for some $a, b \in R$, we must have $b \in (x)$.
		\item Show that $x \mapsto \frac{1}{y} + R$ and $y \mapsto 0$ induces a well-defined $R$-module homomorphism $(x,y) \xrightarrow{\, f \,} Q/R$.
		\item Show that $M$ is a divisible $R$-module, but not injective.
	\end{enumerate}
\end{problem}


\vspace{1em}

\begin{problem}
	$(\star)$ Let $R$ be a domain. Show that if $R$ has a nonzero module $M$ that is both injective and projective, then $R$ must be a field.\footnote{Hint: show that any nonzero $R$-module homomorphism $M \longrightarrow R$ must be surjective, and then show that such a homomorphism must exist.}
\end{problem}

\vspace{2em}

\noindent
\fbox{\begin{minipage}{\textwidth}
An $R$-module $F$ is \emph{faithfully flat} if $F$ is flat and $F \otimes_R M \neq 0$ for every nonzero $R$-module $M$.
\end{minipage}} 

\vspace{0.5em}

\begin{problem} 
$(\star)$ Let $R$ be a commutative ring.
Show that the following are equivalent:	
\begin{enumerate}[a)]
		\item $F$ is faithfully flat.
		\item $F$ is flat and for every proper ideal $I$, $IF \neq F$.
		\item $F$ is flat and for every maximal ideal $\m$, $\m F \neq F$.
		\item The complex
		$$\xymatrix{A \ar[r]^-f & B \ar[r]^-g & C}$$
		is exact if and only if 
		$$\xymatrix{F \otimes_R A \ar[r]^-{1 \otimes f} & F \otimes_R B \ar[r]^-{1 \otimes g} & F \otimes_R C}$$ 
		is exact.
	\end{enumerate}
\end{problem}


\vfill



\begin{problem}
Let $M$ be an $R$-module. Show that $M$ is flat if and only if $\Tor_1^R(M,N) = 0$ for every $R$-module $N$.
\end{problem}



\vfill
	
\begin{problem}
Let $R$ be a commutative ring and $M$ and $N$ be $R$-modules. Consider the $R$-module homomorphism $f\!: M \to M$ given by multiplication by a fixed element $r \in R$. Show that the map $\Ext^i(f,N)\!: \Ext^i_R(M,N) \to \Ext^i_R(M,N)$ induced by $f$ is multiplication by $r$ on $\Ext^i_R(M,N)$.
\end{problem}




\vfill


\newpage

\begin{problem}
Let $(R, \m)$ be a commutative local ring, and let $M$ be a finitely presented $R$-module with minimal presentation
$$\xymatrix{0 \ar[r] & K \ar[r] & R^n \ar[r]^-\pi & M \ar[r] & 0.}$$
Note that the assumption here is that $K$ is also a finitely generated module.
\begin{enumerate}[a)]
	\item Show that if $M$ is flat, then
$$\xymatrix{0 \ar[r] & K \otimes_R R/\m \ar[r] & R^n \otimes_R R/m \ar[r] & M \otimes_R R/m \ar[r] & 0}$$
is exact.
\item Show that 
	  M is free $\iff$ $M$ is projective $\iff$ $M$ is flat.
\end{enumerate}
\end{problem}

\vfill


\begin{problem}
	$(\star)$ Let $R$ be a domain and $Q$ be its fraction field. Let $T$ denote the torsion functor. 
	\begin{enumerate}[a)]
		\item Show that $T(M) = \Tor_1^R(M,Q/R)$.
		\item Show that for every short exact sequence
	$$\xymatrix{0 \ar[r] & A \ar[r] & B \ar[r] & C \ar[r] & 0}$$
	of $R$-modules gives rise to an exact sequence
	$$\xymatrix@C=7mm{0 \ar[r] & T(A) \ar[r] & T(B) \ar[r] & T(C) \ar[r] & (Q/R) \otimes_R A \ar[r] & (Q/R) \otimes_R B \ar[r] & (Q/R) \otimes_R C \ar[r] & 0.}$$
	\item Show that the right derived functors of $T$ are $R^1T = (Q/R) \otimes_R -$ and $R^iT = 0$ for all $i \geqslant 2$.	
	\end{enumerate}
	\end{problem}
	
	

\vfill

\begin{problem}
	Let $k$ be a field, $R = k[x,y]$, and $\m = (x,y)$.
	\begin{enumerate}[a)]
		\item Show that
		$$\xymatrix@C=14mm{0 \ar[r] & R \ar[r]^-{\begin{pmatrix}
			y \\ -x
		\end{pmatrix}} & R^2 \ar[r]^-{\begin{pmatrix}
			x & y
		\end{pmatrix}} & R \ar[r] & 0}$$
		is a free resolution for $k = R/\m$.
		\item Compute $\Tor_i^R(k,k)$ for all $i$.
		\item Show that
		$$\Tor_1(\m,k) \cong \Tor_2(k,k).$$
	\end{enumerate}
\end{problem}
	
\vfill
	
	



\end{document}