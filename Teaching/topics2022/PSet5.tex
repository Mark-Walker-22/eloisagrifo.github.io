\documentclass[11pt]{article}
\usepackage[margin=1in]{geometry}
\usepackage{amsmath,amsfonts,amssymb,amsthm,enumerate}
\usepackage[]{graphicx}
\usepackage{color,subfigure}
\usepackage{multicol}
\usepackage{float}
\usepackage[all]{xypic}
\usepackage{colonequals}
\usepackage{hyperref}
\usepackage{stmaryrd}

\usepackage{fancyhdr, lastpage}
\pagestyle{fancy}
\fancyfoot[C]{{\thepage} of \pageref{LastPage}}

\newtheorem*{theorem}{Theorem}
\newtheorem*{definition}{Definition}


\DeclareMathOperator{\Ass}{Ass}
\DeclareMathOperator{\Min}{Min}
\DeclareMathOperator{\depth}{depth}
\DeclareMathOperator{\pdim}{pdim}
\DeclareMathOperator{\embdim}{embdim}

\newcommand{\m}{\mathfrak{m}}

\title{}
\date{\vspace{-0.5in}}

\makeatletter
\g@addto@macro\@floatboxreset\centering
\makeatother

\theoremstyle{definition}
\newtheorem{problem}{Problem}


\title{Problem Set 0 \\ Introductory Macaulay2 problems}


\begin{document}

\thispagestyle{fancy}
\pagestyle{fancy}
\rhead{UNL $\mid$ Elo\'isa Grifo}
\lhead{Math 918 $\mid$ Spring 2022}


\begin{center}
	{\LARGE Problem Set 5\\
	
%	\vspace{0.5em}
	
%	Symbolic powers
}
\end{center}

\

To solve these problems, you are not allowed to use any additional Macaulay2 packages besides the \texttt{Complexes} package and the ones that are automatically loaded with Macaulay2.


\

\begin{definition}[Jacobian ideal]
	Let $k$ be a field and $R = k[x_1, \ldots, x_n]/I$, where $I = (f_1, \ldots, f_r)$ has pure height $h$. The \emph{Jacobian matrix} of $R$ is the matrix given by
	$$\begin{pmatrix}
		\frac{\partial f_1}{\partial x_1} & \cdots & \frac{\partial f_1}{\partial x_n} \\ & \ddots & \\
		\frac{\partial f_r}{\partial x_1} & \cdots & \frac{\partial f_r}{\partial x_n}
	\end{pmatrix}.$$
	The \emph{jacobian ideal} of $R$ is the ideal generated by the $h$-minors of the Jacobian matrix.
\end{definition}

Turns out the Jacobian ideal is indeed well-defined -- meaning, our definition does not depend on the choice of presentation for $R$ -- and that it determines the nonsmooth locus of $R$. For proofs of these well-known facts, see Section 16.6 of Eisenbud's \emph{Commutative Algebra with a view towards algebraic geometry}.

\begin{theorem}[Jacobian criterion]
	Let $R = k[x_1, \ldots, x_n]/I$ with $k$ a perfect field, and assume that $I$ has pure height $h$. The Jacobian ideal $J$ defines the nonsmooth locus of $R$: a prime $P$ in $k[x_1, \ldots, x_n]$ contains $J$ if and only if $R_P$ is not a regular ring.
\end{theorem}


\

\begin{problem}
	Let $(R, \m)$ be a regular local ring and $I$ be an ideal in $R$.
	\begin{enumerate}[a)]
		\item Show that there is a short exact sequence
		$$\xymatrix{0 \ar[r] & \frac{I+\m^2}{\m^2} \ar[r] & \m/\m^2 \ar[r] & \frac{\m}{\m^2+I} \ar[r] & 0}$$
		and conclude that $\dim_k \left( \frac{I+\m^2}{\m^2}\right) = \embdim(R) - \embdim(R/I)$.
		\item Show that if $R/I$ is regular, then there exists a minimal set of generators $x_1, \ldots, x_d$ for $\m$ such that $I = (x_1, \ldots, x_n)$ for some $n$.
		\item Conclude that if $I$ is not generated by a regular sequence, then $R/I$ is not regular.
	\end{enumerate}
\end{problem}


\begin{problem}
	Let $I$ be a radical ideal in $R = k[x_1, \ldots, x_n]$, where $k$ is a perfect field, and let $J$ be the Jacobian ideal of $I$. Prove that $I^{(n)} = (I^n : J^\infty)$ for all $n \geqslant 1$.
\end{problem}


\begin{problem}
	 Let $k$ be a field, $R = k[x,y,z]$, and $I = (xy,xz,yz)$. Find the Jacobian ideal $J$ of $I$, and show directly that $I^{(n)} = (I^n : J^\infty)$ for all $n \geqslant 1$ without using Problem 2.
\end{problem}


\begin{problem}
	For each of the following ideals $I$, find an {\bf element} $t$ such that $I^{(n)} = (I^n : t^\infty)$ for all $n \geqslant 1$.
	\begin{enumerate}[a)]
		\item $I = (xz,xw,yz,yw)$ in $R = k[x,y,z,w]$, where $k$ is any field.
		\item $I$ is the defining ideal of the second Veronese in $3$ variables $\mathbb{Q}[x,y,z]^{(2)}$.
	\end{enumerate}
\end{problem}

\newpage

\begin{problem}
	Let $k$ be a field, $R = k[x,y,z]$, and $I = (xy,xz,yz)$.
	\begin{enumerate}[a)]
		\item Show that $I^{(2n)} = \left( I^{(2)} \right)^n$ for all $n \geqslant 1$.
		\item Show that $I^{(2n+1)} = I \left( I^{(2)} \right)^n$ for all $n \geqslant 1$.
		\item Show that the symbolic Rees algebra of $I$ is a noetherian ring.
	\end{enumerate}
\end{problem}



\begin{problem}
	Let $R \to S$ be a flat ring homomorphism.
	\begin{enumerate}[a)]
		\item Show that for every ideal $I$ in $R$ and every $x \in R$, $(I :_R x) S = (IS :_S x)$. 
		
	Hint: what is the kernel of $R/I \xrightarrow{\cdot x \, } R/I$?
		
		\item Show that $I$ and $J$ are ideals in $R$, then $IS \cap JS = (I \cap J)S$.
		
	Hint: what is the kernel of the canonical map $R \to R/I \oplus R/J$?	
		
		\item Show that if $I$ and $J$ are ideals in $R$ with $J$ finitely generated, then $(I :_R J)S = (IS :_S JS)$.
		
	\end{enumerate}
\end{problem}


\begin{problem}
	Let $I$ be a squarefree monomial ideal in $R = k[x_1, \ldots, x_d]$, where $k$ is a perfect field of prime characteristic $p$.
	\begin{enumerate}[a)]
		\item Show that $R/I$ is $F$-pure.
		\item Show that $R/I$ is strongly $F$-regular if and only if $I$ is generated by variables.
	\end{enumerate}
\end{problem}








\end{document}