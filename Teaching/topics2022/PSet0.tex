\documentclass[11pt]{article}
\usepackage[margin=1in]{geometry}
\usepackage{amsmath,amsfonts,amssymb,amsthm,enumerate}
\usepackage[]{graphicx}
\usepackage{color,subfigure}
\usepackage{multicol}
\usepackage{float}
\usepackage[all]{xypic}
\usepackage[colorlinks=false,citecolor=cyan,linkcolor=magenta]{hyperref}

\usepackage{fancyhdr, lastpage}
\pagestyle{fancy}
\fancyfoot[C]{{\thepage} of \pageref{LastPage}}


\title{}
\date{\vspace{-0.5in}}

\makeatletter
\g@addto@macro\@floatboxreset\centering
\makeatother

\theoremstyle{definition}
\newtheorem{problem}{Problem}


\title{Problem Set 0 \\ Introductory Macaulay2 problems}


\begin{document}

\thispagestyle{fancy}
\pagestyle{fancy}
\rhead{UNL $\mid$ Elo\'isa Grifo}
\lhead{Math 918 $\mid$ Spring 2022}


\begin{center}
	{\LARGE Problem Set 0\\
	
	\vspace{0.5em}
	
	Introductory Macaulay2 problems}
\end{center}

\


\begin{problem} $\,$
\begin{enumerate}[a)]
	\item Install Macaulay2.\footnote{If you don't have access to a computer, or if your computer runs only Windows, come talk to me about it.} Hardcore version: install emacs and run Macaulay2 through emacs.
	\item Make an .m2 file setting up a field $k$, a polynomial ring $R$ over $k$, a nontrivial ideal $I$ in $R$, the $R$-module $M = R/I$ and the ring $S = R/I$.
\end{enumerate}
\end{problem}




\begin{problem}[Subalgebras]
	Use Macaulay2 to find:
	\begin{enumerate}[a)]
		\item A presentation for the $\mathbb{Q}$-algebra $\mathbb{Q}[xy,xu,yv,uv] \subseteq \mathbb{Q}[x,y,u,v]$.
		\item A presentation for the $k$-algebra $U$, where $k = \mathbb{Z}/101$ and
		$$k \begin{bmatrix} ux & uy & uz \\ vx & vy & vz \end{bmatrix} \subseteq \frac{k[u,v,x,y,z]}{(x^3+y^3+z^3)}.$$
	\end{enumerate}
\end{problem}



\begin{problem}[Graded rings]
$\,$
	\begin{enumerate}[a)]
		\item In Macaulay2, set up $A = \mathbb{Q}[s^2,st,t^2]$ as an $\mathbb{N}^2$-graded ring with the grading induced by setting $s^2, st, t^2$ as homogeneous elements of degrees
		$$\deg(s^2) = (2,0) \quad \deg(st) = (1,1) \quad \deg(t^2) = (0,2).$$
		\item The ring $R=\mathbb{Q}[t^3,t^{13},t^{42}]$ is a graded subring of $\mathbb{Q}[t]$ with the standard grading, meaning that the graded structure on $\mathbb{Q}[t]$ induces a grading on $R$. Set up $R$ (with this grading) in Macaulay2.
	\end{enumerate}
\end{problem}



\begin{problem}[Modules]
	Consider the domain $R = \mathbb{Q}[x,y,z,a,b,c]/(xb-ac,yc-bz,xc-az)$. Set up the following $R$-modules, making sure Macaulay2 actually sees them as modules over $R$:
	\begin{enumerate}[a)]
		\item The ideal $I = (x,a)$ viewed as an $R$-module.
		\item The $R$-module $N = \mathbb{Q}$.
		\item The $2$-generated $R$-module $M = Rf + Rg$, where the generators $f, g$ satisfy the relations 
		$$yf-xg = 0 \quad bf - cg = 0 \quad cf - zg = 0.$$
		\item The submodule of $R^3$ generated by $(a,b,c)$ and $(x,y,z)$.
	\end{enumerate}
\end{problem}




%
%
%\begin{problem}
%Height and dimension.
%	\begin{enumerate}[a)]
%		\item Find the height of $J = (ab,bc,cd,ad)$ in $k[a,b,c,d]$ over any field $k$, and the dimension of $k[a,b,c,d]/J$.
%		\item Find the dimension of the ring $S = \mathbb{Q}[x^3y^3, x^3y^2z, x^2z^3] \subseteq \mathbb{Q}[x,y,z]$.
%		\item Let $I$ be the defining ideal of the curve parametrized by $(t^{13},t^{42},t^{73})$ over $\mathbb{Q}$. Find the height of $I$, and notice that $\textrm{height}(I) < \mu(I)$.
%		\item Let $R=\mathbb{Q}[x,y,z]$, and $I=(x^3, x^2y, x^2z, xyz)$. Find the dimension of $R/I$ and the height of $I$.
%		\item Find the dimension of the module $I/I^2$, where $I = (xz)$ in $R = \mathbb{C}[x,y,z]/(xy,yz)$.	
%	\end{enumerate}
%\end{problem}
%
%
%
%\begin{problem}[Minimal generators]
%$\,$
%	\begin{enumerate}[a)]
%			\item Let $I$ be the defining ideal of the curve parametrized by $(t^{13},t^{42},t^{73})$ over $\mathbb{Q}$. Find $\mu(I)$ and a minimal generating set for $I$.
%	\end{enumerate}
%\end{problem}
%
%


	
\begin{problem}[Complexes in Macaulay2]
	Let $R = \mathbb{Q}[x,y,z]/(x^2,xy)$. 
	
\begin{enumerate}[a)]
\item Consider the bounded complex
	$$C = \xymatrix@R=1mm@C=30mm{R \ar[r]^-{\begin{pmatrix} z \\ -y \\ x \end{pmatrix}} & R^3 \ar[r]^-{\begin{pmatrix} -y & -z & 0 \\ x & 0 & -z \\ 0 & x & y \end{pmatrix}
} & R^3 \ar[r]^-{\begin{pmatrix} x & y & z\end{pmatrix}
} & R \\ \text{{\tiny 3}} & \text{{\tiny 2}} & \text{{\tiny 1}} & \text{{\tiny 0}}}$$
Set $C$ up in Macaulay2 and compute its homology. For which $n$ is $\textrm{H}_n(C) = 0$?
\item Check that $f$ below is a map of complexes, and compute its kernel, cokernel, and homology.
	$$\xymatrix@R=2mm{D = \\ \\ \\ \\ \\ C = \ar@<1ex>[uuuuu]^-f} \xymatrix@R=2mm@C=30mm{R \ar[r]^-{\begin{pmatrix} z \\ -y \\ x \end{pmatrix}} & R^3 \ar[r]^-{\begin{pmatrix} -y & -z & 0 \\ x & 0 & -z \\ 0 & x & y \end{pmatrix}
} & R^3 \ar[r]^-{\begin{pmatrix} x & y & z\end{pmatrix}
} & R \\ &&&\\ &&&\\ &&&\\ &&&\\ 0 \ar[r]_-0 \ar[uuuuu]^-{0} & R \ar[uuuuu]^-{\begin{pmatrix} 0 \\ 0 \\ 1 \end{pmatrix}} \ar[r]_-{\begin{pmatrix} -z \\ y \end{pmatrix}}  \ar[uuuuu]^-{} & R^2 \ar[uuuuu]^-{\begin{pmatrix} 0 & 0 \\ 1 & 0 \\ 0 & 1 \end{pmatrix}} \ar[r]_-{\begin{pmatrix} y & z\end{pmatrix}} & R. \ar@{=}[uuuuu] \\
\text{{\tiny 3}} & \text{{\tiny 2}} & \text{{\tiny 1}} & \text{{\tiny 0}}}$$
\end{enumerate}
\end{problem}

\end{document}