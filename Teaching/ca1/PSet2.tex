\documentclass[11pt]{article}
\usepackage[margin=1in]{geometry}
\usepackage{amsmath,amsfonts,amssymb,amsthm,enumerate}
\usepackage[]{graphicx}
\usepackage{color,subfigure}
\usepackage{multicol}
\usepackage{float}
\usepackage[all]{xypic}
\usepackage[colorlinks=true,citecolor=cyan,linkcolor=magenta]{hyperref}
\usepackage{colonequals}

\usepackage{fancyhdr, lastpage}
\pagestyle{fancy}
\fancyfoot[C]{{\thepage} of \pageref{LastPage}}


\title{}
\date{\vspace{-0.5in}}

\makeatletter
\g@addto@macro\@floatboxreset\centering
\makeatother

\theoremstyle{definition}
\newtheorem{problem}{Problem}


\title{Homework 1}

\begin{document}

\thispagestyle{fancy}
\pagestyle{fancy}
\rhead{UNL $\mid$ Elo\'isa Grifo}
\lhead{Commutative Algebra I Fall 2022}

\vspace{3em}

\begin{center}
	{\LARGE Problem Set 2}
\end{center}

\vspace{1em}

\noindent
{\bf Instructions:}
%For full credit, turn in 5 problems, split between a pdf and a .m2 file (one of each). 
You are welcome to work together with your classmates on all the problems, and I will be happy to give you hints or discuss the problems with you, but you should write up your solutions by yourself.
You cannot use any resources besides me, your classmates, our course notes, and the Macaulay2 documentation.


\vspace{2em}
%
%\begin{problem}
%	In Macaulay2, set up a polynomial ring $R$ over a field $k$, a nontrivial ideal $I$ in $R$, the $R$-module $M = R/I$ and the ring $S = R/I$. Your work must be turned in as (part of) a .m2 file.
%\end{problem}


\noindent
\fbox{\begin{minipage}{\textwidth}

Let $R$ be a ring and $M$ an $R$-module. The {\bf Nagata idealization} of $(R,M)$ is the ring $R \rtimes M$ defined as follows:
\begin{itemize}
\item as a set, $R \rtimes M = R\times M$;
\item the addition is $(r,m) + (s,n)=(r+s,m+n)$;
\item the multiplication is $(r,m)  (s,n)=(rs,sm+rn)$.
\end{itemize}

Then $R \rtimes M$ with the operations specified above is a ring.

\

Note that $R$ is a subring of $R\rtimes M$ (via the inclusion $r\mapsto (r,0)$), and as an $R$-module, $R\rtimes M\cong R\oplus M$.
\end{minipage}} 

\


\begin{problem}
In this problem, we will construct an extension of rings $A \subseteq B \subseteq C$ such that $A \subseteq C$ is module-finite, but $A \subseteq B$ is not.
%\footnote{We remarked before that such examples exist, but we didn't construct one.}

\begin{enumerate}[a)]
\item Can you find such an extension with $A$ noetherian?
\item Let $R$ be a ring that is not noetherian, and $I$ an ideal that is not finitely generated. 
Show that ${R \subseteq R \rtimes I \subseteq R \rtimes R}$, that $R \subseteq R \rtimes R$ is module-finite, but  $R \subseteq R \rtimes I$ is not.
\end{enumerate}
\end{problem}




\begin{problem}
	Suppose that every ascending chain of prime ideals in $R$ stabilizes. Must $R$ be a noetherian ring? 
\end{problem}



\begin{problem}
Show that $R$ is a noetherian ring if and only if every prime ideal of $R$ is finitely generated.
Spoiler alert: there's a hint in the footnotes.\footnote{Hint: use Zorn's Lemma to show that if $R$ is a ring that is not noetherian, the set of ideals that are not finitely generated has a maximal element; show that element is a prime ideal.

Second hint: take an ideal $I$ that is maximal among the ideals that are not finitely generated, take $f,g\notin I$ with $fg\in I$, and consider the ideals $(I,f)$ and $(I:f)  \colonequals \{ r\in R \ | \ rf\in I\}$.}
\end{problem}


\begin{problem}
	Let $k$ be a field. Show that any $k$-algebra $R$ with $k \subseteq R \subseteq k[x]$ is algebra-finite over~$k$.
\end{problem}


\begin{problem}[Graded rings]
$\,$
	\begin{enumerate}[a)]
		\item In Macaulay2, set up $A = \mathbb{Q}[s^3,s^2t,st^2,t^3]$ as an $\mathbb{N}^2$-graded ring with the grading induced by setting $s^3, s^2t, st^2, t^3$ as homogeneous elements of degrees
		$$\deg(s^3) = (3,0) \quad \deg(s^2t) = (2,1) \quad \deg(st^2) = (1,2) \quad \deg(t^3) = (0,3).$$
	Check that these $4$ generators have the desired degrees and that your ring is graded.
		\item The ring $R=\mathbb{Q}[t^3,t^{13},t^{47}]$ is a graded subring of $\mathbb{Q}[t]$ with the standard grading, meaning that the graded structure on $\mathbb{Q}[t]$ induces a grading on $R$. Set up $R$ (with this grading) in Macaulay2.
		Check that the $3$ algebra generators have the desired degrees and that your ring is graded.
	\end{enumerate}
\end{problem}



\end{document}