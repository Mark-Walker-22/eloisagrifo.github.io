\documentclass[11pt]{article}
\usepackage[margin=1in]{geometry}
\usepackage{amsmath,amsfonts,amssymb,amsthm,enumerate}
\usepackage[]{graphicx}
\usepackage{color,subfigure}
\usepackage{multicol}
\usepackage{float}
\usepackage[all]{xypic}
\usepackage[colorlinks=true,citecolor=cyan,linkcolor=magenta]{hyperref}
\usepackage{colonequals}

\usepackage{fancyhdr, lastpage}
\pagestyle{fancy}
\fancyfoot[C]{{\thepage} of \pageref{LastPage}}


\DeclareMathOperator{\mSpec}{mSpec}
\DeclareMathOperator{\Spec}{Spec}
\DeclareMathOperator{\Ass}{Ass}
\DeclareMathOperator{\Supp}{Supp}
\DeclareMathOperator{\height}{height}

\newcommand{\m}{\mathfrak{m}}
\newcommand{\C}{\mathbb{C}}

\title{}
\date{\vspace{-0.5in}}

\makeatletter
\g@addto@macro\@floatboxreset\centering
\makeatother

\theoremstyle{definition}
\newtheorem{problem}{Problem}


\title{Homework 1}

\begin{document}

\thispagestyle{fancy}
\pagestyle{fancy}
\rhead{UNL $\mid$ Elo\'isa Grifo}
\lhead{Commutative Algebra I Fall 2022}

\vspace{3em}

\begin{center}
	{\LARGE Problem Set 5}
\end{center}

\

\noindent
{\bf Instructions:}
For full credit, turn in 4 problems in a pdf file and a .m2 file. 
You are welcome to work together with your classmates on all the problems, and I will be happy to give you hints or discuss the problems with you, but you should write up your solutions by yourself.
You cannot use any resources besides me, your classmates, our course notes, and the Macaulay2 documentation.


\vspace{2em}


\begin{problem}
	Let $R$ be a noetherian ring and let $I$ be a proper ideal in $R$.
	\begin{enumerate}[a)]
		\item Show that there exists an integer $n$ such that $(\sqrt{I})^n \subseteq I$.
%		\item Let $\m J = Q_1 \cap \cdots \cap Q_s$ be a primary decomposition. Show that $J \subseteq Q_i$ for each $i$.
%		\item Prove that $J = (0)$.
		\item Show that if $R$ is local, then
		$$\bigcap_{n \geqslant 0} I^n = (0).$$
		
		\noindent
		Hint: reduce to the case when $I = \m$, the unique maximal ideal. In that case, let $J \colonequals \displaystyle\bigcap_{n\geqslant 0} \m^n$, and show that if $\m J = Q_1 \cap \cdots \cap Q_s$ is a primary decomposition, then $J \subseteq Q_i$ for all $i$.
		
		\item Show that if $R$ is a domain, then
		$$\bigcap_{n \geqslant 0} I^n = (0).$$
		\item What if $R$ is not local nor a domain?
	\end{enumerate}
\end{problem}



\vfill



\begin{problem}
For each of the following rings $R$ determine whether the inclusion $\C[x] \subseteq R$ is integral, with justification. 
\begin{enumerate}[a)]
\item $R=\C[x,y,z]/(z^2-xy)$
\item $R=\C[x,y,z]/(y^3-x^2, z^2-xy)$
\item $R=\C[x,y,z]/(x^2-yz, y^2-xz, z^2-xy)$
\end{enumerate}
If the answer is no, you might find it helpful to consider properties of the induced map on spectra.
\end{problem}

\vfill


\begin{problem}
	For each of the following rings, compute their dimension in Macaulay2 for some particular field $k$ of your choice, and then prove that the answer is correct for all fields $k$.

\begin{enumerate}[a)]
\item $\displaystyle R=\frac{k[x,y,z]}{(x^3,x^2y,xyz)}$.
\smallskip
\item $\displaystyle R=\frac{k[x,y,z,u,v]}{(x^3 u^2 + y^3 uv + z^3 v^2)}$.
\smallskip
\item $\displaystyle R=k[x^2 u, xyu, y^2 u, x^2 v, xyv, y^2v] \subseteq k[x,y,u,v]$.
%\smallskip
%\item $\displaystyle R=\frac{k[x,y,u,v]}{(u^3 - xy, v^5 - x^2 u - y^3)}$.
\smallskip
\item $R= k [x,y,z]/(x^2-yz, y^2-xz, z^2-xy)$
\end{enumerate}

\end{problem}


\vfill

\begin{problem}
	Let $k$ be a field and $R = k[a,b,c,d]/(ad-bc)$. Find prime ideals $P$ and $Q$ such that $\height(P) + \height(Q) < \height(P+Q)$, or show that no such primes exist.
\end{problem}

\newpage


\begin{problem}
	To answer the following questions, you can use Macaulay2 to help, but you are also required to explain your answers in the pdf file; you do not need to prove that the results of your Macaulay2 calculations are correct. You will likely get a warning about computations over $\mathbb{C}$ being inexact; nevertheless, you can believe the results of your calculation. Unfortunately, some calculations you might try to do will not work -- but they won't prevent you from answering these questions.
	\begin{enumerate}[a)]
		\item Let $S = \mathbb{C}[x,y]/(xy^3,x^3y)$, $P = (x,y+1)/(xy^3,x^3y)$, and consider the $S$-submodule $N$ of $S^2$ generated by $(x,y)$ and $(y,x)$. Is $N_P = 0$?
		\item Let $R = \mathbb{C}[x,y,z]/(xy,yz)$ and consider the $R$-module $M = I/I^2$, where $I = (xz)$. Explicitly describe $\Supp(M)$.
	\end{enumerate}
\end{problem}

%\
%
%
%\begin{problem}
%Let $X= \begin{bmatrix} x_{11} & x_{12} & x_{13} \\ x_{21} & x_{22} & x_{23} \end{bmatrix}$ be a $2\times 3$ matrix of indeterminates over $\mathbb{C}$ and $R=\mathbb{C}[X]$. Let $I$ be the ideal of $2\times 2$ minors of $X$. 
%	\begin{enumerate}[a)]
%		\item Show that $S = \mathbb{C}[x_{11}, x_{12}-x_{21}, x_{13}-x_{22}, x_{23}]$ is a Noether normalization for $R/I$.
%		
%		Hint 1: to show that $S \subseteq R/I = M$ is module-finite, consider $M/S_+M$.
%		
%		Hint 2: to show that $S \subseteq R/I$, consider the quotient map $R/I \to \mathbb{C}[]$
%		
%		\item Find $\height(I)$.
%	\end{enumerate}
%\end{problem}

\

\begin{problem}
	For each of the following ideals, compute their height in Macaulay2 for some particular field $k$ of your choice, and then prove that the answer is correct for all fields $k$.

\begin{enumerate}[a)]
\item $I = (a^2,ab,bc,cd,d^2)$ in $k[a,b,c,d]$.
\item $I = (x,z)$ in $k[x,y,z]/(xy-z^2)$.
\item The kernel $I$ of the ring homomorphism $f\!: k[a,b,c,d,e] \to k[x,y]$ given by
$$f(a) = x^4 \quad f(b) = x^3y \quad f(c) = x^2y^2 \quad f(d) = xy^3 \quad f(e) = y^4.$$
\end{enumerate}

\end{problem}



\end{document}