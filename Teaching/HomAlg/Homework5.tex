\documentclass[11pt]{article}
\usepackage[margin=1in]{geometry}
\usepackage{amsmath,amsfonts,amssymb,amsthm,mathrsfs}
\usepackage[]{graphicx}
\usepackage{color,subfigure}
\usepackage{multicol}
\usepackage{enumerate}
\usepackage{colonequals}
\usepackage{float}
\usepackage[all]{xypic}
\usepackage[colorlinks=true,citecolor=cyan,linkcolor=magenta]{hyperref}
\usepackage{stmaryrd}
\usepackage[bottom]{footmisc}



\usepackage{fancyhdr, lastpage}
\pagestyle{fancy}
\fancyfoot[C]{{\thepage} of \pageref{LastPage}}


\DeclareMathOperator{\Ch}{Ch}
\DeclareMathOperator{\Hom}{Hom}
\DeclareMathOperator{\coker}{coker}
\DeclareMathOperator{\im}{im}
\DeclareMathOperator{\Tor}{Tor}
\DeclareMathOperator{\Ext}{Ext}
\DeclareMathOperator{\depth}{depth}
\DeclareMathOperator{\pdim}{pdim}

\newcommand{\m}{\mathfrak{m}}
\newcommand{\n}{\mathfrak{n}}

\title{}
\date{\vspace{-0.5in}}

\makeatletter
\g@addto@macro\@floatboxreset\centering
\makeatother

\theoremstyle{definition}
\newtheorem{problem}{Problem}


\title{Problem Set 1}

\begin{document}

\thispagestyle{fancy}
\pagestyle{fancy}
\rhead{UCR $\mid$ Elo\'isa Grifo}
\lhead{Homological Algebra Spring 2021}


\begin{center}
	{\LARGE Problem Set 5}
\end{center}


\begin{problem}
	Let $R$ be a domain and $Q$ be its fraction field. Let $T(-)$ denote the torsion functor we introduced in Problem Set 3. 
	\begin{enumerate}[a)]
		\item Show that $T(M) = \Tor_1^R(M,Q/R)$.\footnote{Hint: you want to look at some long exact sequence for Tor.}
		\item Show that for every short exact sequence
	$$\xymatrix{0 \ar[r] & A \ar[r] & B \ar[r] & C \ar[r] & 0}$$
	of $R$-modules gives rise to an exact sequence\footnote{Hint: apply the Snake Lemma to some nice diagram.}
	$$\xymatrix@C=7mm{0 \ar[r] & T(A) \ar[r] & T(B) \ar[r] & T(C) \ar[r] & (Q/R) \otimes_R A \ar[r] & (Q/R) \otimes_R B \ar[r] & (Q/R) \otimes_R C \ar[r] & 0.}$$
	\item Show that the right derived functors of $T$ are $R^1T = (Q/R) \otimes_R -$ and $R^iT = 0$ for all $i \leqslant 2$.	
	\end{enumerate}
	\end{problem}
	
	\vfill


\begin{problem}
	Let $I$ be an ideal in $R$. Show that
	$$\Ext^{n}_R(I,M) \cong \Ext^{n+1}_R(R/I,M)$$
	for all $n \geqslant 1$ and all $R$-modules $M$.
\end{problem}

\vfill

\begin{problem}
	Let $(R,\m)$ be a Noetherian local ring. Let $r \in R$ and $M$ and $N$ be finitely generated $R$-modules.
	\begin{enumerate}[a)]
		\item Show that the map $\Ext^i_R(M,N) \to \Ext^i_R(M,N)$ induced by $\xymatrix{M \ar[r]^-r & M}$ is the map given by multiplication by $r$.
		\item Show that if $r$ is regular on $M$ and $\Ext^i_R(M/rM,N) = 0$ for $i \gg 0$, then $\Ext^i_R(M,N) = 0$ for $i \gg 0$.
	\end{enumerate}
\end{problem}

\vfill

\begin{problem}
	Let $(R,\m)$ be a Noetherian local ring.
	\begin{enumerate}[a)]
		\item Show that for every short exact sequence $\xymatrix{0 \ar[r] & A \ar[r] & B \ar[r] & C \ar[r] & 0}$ of $R$-modules,
		$$\depth(A) \geqslant \min \lbrace \depth(B), \depth(C) + 1 \rbrace.$$
		\item Given any finitely generated $R$-module $M$, show that there exists $n \geqslant 1$ such that either $\pdim(M) < n$ or $\depth(\Omega_n M) = \depth R$.
	\end{enumerate}
\end{problem}


\newpage

\setcounter{problem}{6}

\begin{problem}
	Consider the ring $R = \mathbb{Q}[x,y,z,a,b,c]/(xb-ac,yc-bz,xc-az)$ and the $2$-generated $R$-module $M = Rf + Rg$, where the generators $f, g$ satisfy the relations 
		$$yf-xg = 0 \quad bf - cg = 0 \quad cf - zg = 0.$$
		Let $P$ be the ideal in $S = \mathbb{Q}[x,y,z]$ defining the curve $\lbrace (t^{13},t^{42},t^{73}) \mid t \in \mathbb{Q} \rbrace$.

\

To solve this problem, you are not allowed to use any additional Macaulay2 packages besides the \texttt{Complexes} package and the ones that are automatically loaded with Macaulay2.

\begin{enumerate}[a)]
	\item Find $\pdim_S(S/P)$ and $\depth(S/P)$. 
	\item Is $P$ generated by a regular sequence?
	\item Find $\pdim_R(M)$ and $\depth(M)$.
	\item Is $R$ a regular ring? Is it Cohen-Macaulay?
\end{enumerate}
\end{problem}



\end{document}